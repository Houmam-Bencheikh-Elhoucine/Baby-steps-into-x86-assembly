\documentclass[10pt]{article}
\usepackage[utf8]{inputenc}
\usepackage{tabto, tabularx}
\title{\centering\textbf{Baby steps into x86 assembly language}}
\author{Hamame BENCHEIKH LEHOCINE}
\date{\today}

\begin{document}
    
    \maketitle
    \newpage
    \begin{abstract}
        \centering
        This document is an introduction to x86 assembly addressed to computer science students 
        to help them understand the different instructions an x86 CPU can perform and to help them 
        write and read assembly code and get an idea about its output.


        It will guide the reader from basic components of the CPU to the x86 instruction set
        and how they work and finally the binary encoding of the assembly code.
    \end{abstract}
    \newpage
    \section{Inside of the 80\_86 processor : }
    \subsection{What is a CPU : }
    \tabto{20pt}The CPU (central processing unit) is the most important component of the computer,
     it takes charge of executing all the instructions and operations the computer has to perform.

    It is resposible on performing all math and logic operations 
    and comunicate with other components like the central memory 
    (RAM) and external devices (Hard drive, Monitor, keyboard ...) 
    as it handles any errors caused by hardware or software.
    
    \textbf{Note: }the word CPU is used to refer to a single CPU core.

    \subsection{Components of an x86 CPU : }
    The CPU has some specific components and electrical circuits to help it do the job correctly and fast.
    \subsubsection{Clock : }
    The clock is what organizes the CPU work.The CPU stays synchronized to it as it only executes an
     operation within a time interval called a cycle then it moves to the second one, with that nothing will be executed in a mess.
    
    The higher the clock frequency, the faster the processor performs its actions.
    
    Actually, CPU clock speed ranges between 1.5 and more than 5 GHrz of clock frequency, 
    meaning the processor can perform between 4 Billion to 5 Billion operations per second.
    \textbf{fun fact: }The fastest clock speed ever reached is 8.805 Ghrz on an AMD Fx-8150 overclocked.

    \subsubsection{Registers :}
    Registers are chunks of memory with very low size but they work as fast as the clock speed. 
    they are used for storing data fetched from memory, 
    saving information for very short range usage and also important data like the adress of the upcoming instruction and memory sections (more on that later).
    the very old CPUs had only 16 bits = 2 bytes registers, then 32 bits = 4 bytes registers, and recent ones have 
    64 bits = 8 bytes registers.

    in this document we will focus on 16 bit and 32 bit registers.
    \paragraph{General Purpose Registers: }As the name suggests, they serve multiple purposes.
    These registers are accessible to the user so he can modify their contents in his program.
    They can be used for anything but generally they are used for specific operations.(they will be explained later)
    In an x86 16 bit CPU, there are 8 GPRs: 
    \begin{itemize}
        \item \textbf{Accumulator Register (AX):}  used for arithmetic and logic operations holding one of the operands or holding the result of the calculation.
        \item \textbf{Base Register (BX):}  used to store the offset in the RAM.
        \item \textbf{Counter Register (CX):}  used in instructions that require repitition.
        \item \textbf{Data Register (DX):}  used when there is not enough space in the AX register.
        \item \textbf{stack Base Pointer Register (BP):}  used to determine the adress of the base of the stack.
        \item \textbf{Stack Pointer Register (SP):}  used to determine the adress of the top of the stack.
        \item \textbf{Source Index Register (SI):}  used in string operations holding the adress of the source string.
        \item \textbf{Destination Index Register (DI):}  used in string operations holding the adress of the destination string.
    \end{itemize}
    These registers are able to be devided into parts of half their original size. as they have been extended in 32 bit processors.

    the diagram below shows how the registers are devided
    %TODO: make the graphic%
    \paragraph{Section Registers: }These pointers determine the begining of each section in the central memory, 
    following the Von Neueman architecture
    These registers can not be modified in the code.
    there are 4 memory sections:
    \begin{itemize}
        \item \textbf{Data Section(DS): }determines begining of the section where all data is stored.
        \item \textbf{Code Section(CS): }determines begining of the section where the programs are stored.
        \item \textbf{Stack Section(SS): }determines the begining of the stack section.
        \item \textbf{Extra Section(ES): }determines the begining of the extra section.
    \end{itemize}
    There can be more registers to determine more extra sections like FS and GS but we won't care about them.

    The memory sections that we will care about are Data section, Code section and stack section.

    \paragraph{Instruction Pointer and Instruction Register: }The instruction pointer register saves the adress of the next instruction to run.
    The instruction register saves the instruction to execute once loaded from the central memory.
\end{document}