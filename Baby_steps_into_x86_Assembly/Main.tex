\documentclass[10pt]{article}
\usepackage[utf8]{inputenc}
\usepackage{tabto, hyperref}
\title{\centering\textbf{Baby steps into x86 assembly language}}
\author{Hamame BENCHEIKH LEHOCINE}
\date{\today}

\begin{document}
    
    \maketitle
    \newpage
    \begin{abstract}
        \centering
        This document is an introduction to x86 assembly addressed to computer science students 
        to help them understand the different instructions an x86 CPU can perform and to help them 
        write and read assembly code and get an idea about its output.


        It will guide the reader from basic components of the CPU to the x86 instruction set
        and how they work and finally the binary encoding of the assembly code.
    \end{abstract}
    \newpage
    \section{Inside of the 80\_86 processor : }
    \subsection{What is a CPU : }
    \tabto{20pt}The CPU (central processing unit) is the most important component of the computer,
     it takes charge of executing all the instructions and operations the computer has to perform.

    It is resposible on performing all math and logic operations 
    and comunicate with other components like the central memory 
    (RAM) and external devices (Hard drive, Monitor, keyboard ...) 
    as it handles any errors caused by hardware or software.
    
    \textbf{Note: }the word CPU is used to refer to a single CPU core.

    \subsection{Components of an x86 CPU : }
    \tabto{20pt}The CPU has some specific components and electrical circuits to help it do the job correctly and fast.
    \subsubsection{Clock : }
    \tabto{20pt}The clock is what organizes the CPU work.The CPU stays synchronized to it as it only executes an
     operation within a time interval called a cycle then it moves to the second one, with that nothing will be executed in a mess.
    
    The higher the clock frequency, the faster the processor performs its actions.
    
    Actually, CPU clock speed ranges between 1.5 and more than 5 GHrz of clock frequency, 
    meaning the processor can perform between 4 Billion to 5 Billion operations per second.
    \\\textbf{Fun fact: }The fastest clock speed ever reached is 8.805 Ghrz on an AMD Fx-8150 overclocked.

    \subsubsection{Registers :}
    \tabto{20pt}Registers are chunks of memory with very low size but they work as fast as the clock speed. 
    they are used for storing data fetched from memory, 
    saving information for very short range usage and also important data like the adress of the upcoming instruction and memory sections (more on that later).
    the very old CPUs had only 16 bits = 2 bytes registers, then 32 bits = 4 bytes registers, and recent ones have 
    64 bits = 8 bytes registers.

    in this document we will focus on 16 bit and 32 bit registers.
    \paragraph{General Purpose Registers: }As the name suggests, they serve multiple purposes.
    These registers are accessible to the user so he can modify their contents in his program.
    They can be used for anything but generally they are used for specific operations.(they will be explained later)
    In an x86 16 bit CPU, there are 8 GPRs: 
    \begin{itemize}
        \item \textbf{Accumulator Register (AX):}  used for arithmetic and logic operations holding one of the operands or holding the result of the calculation.
        \item \textbf{Base Register (BX):}  used to store the offset in the RAM.
        \item \textbf{Counter Register (CX):}  used in instructions that require repitition.
        \item \textbf{Data Register (DX):}  used when there is not enough space in the AX register.
        \item \textbf{stack Base Pointer Register (BP):}  used to determine the adress of the base of the stack.
        \item \textbf{Stack Pointer Register (SP):}  used to determine the adress of the top of the stack.
        \item \textbf{Source Index Register (SI):}  used in string operations holding the adress of the source string.
        \item \textbf{Destination Index Register (DI):}  used in string operations holding the adress of the destination string.
    \end{itemize}
    These registers are able to be devided into parts of half their original size. as they have been extended in 32 bit processors.

    the diagram below shows how the registers are devided
    %TODO: make the graphic%
    \paragraph{Section Registers: }These pointers determine the begining of each section in the central memory, 
    following the Von Neueman architecture
    These registers can not be modified in the code.
    there are 4 memory sections:
    \begin{itemize}
        \item \textbf{Data Section(DS): }determines begining of the section where all data is stored.
        \item \textbf{Code Section(CS): }determines begining of the section where the programs are stored.
        \item \textbf{Stack Section(SS): }determines the begining of the stack section.
        \item \textbf{Extra Section(ES): }determines the begining of the extra section.
    \end{itemize}
    There can be more registers to determine more extra sections like FS and GS but we won't care about them.

    The memory sections that we will care about are Data section, Code section and stack section.

    \paragraph{Instruction Pointer and Instruction Register: }The instruction pointer register saves the adress of the next instruction to run.
    The instruction register saves the instruction to execute once loaded from the central memory.
    \newpage
    
    \section{Hello assembly : }
    \tabto{20pt}In this section we will discuss the basic x86 assembly instructions and how each one works.
    
    But first, what is assembly language??
    \subsection{What is assembly language : }
    \tabto{20pt}Basically, assembly language is the simplified form of machine code, invented in %year of invention
    by %Creator of assembly language
    to help developpers write better code so that instead of ones and zeros in binary code, we write actual english-like words to refer to instructions and
    all the other stuff.

    Assembly language depends on the processor architecture and the operating system used, x86 instructions are 
    different from ARM instructions and Windows assembly is different from Linux or MAC OS assembly (in some parts).

    Here we will be writing assembly for x86 CPU (Intel or AMD) on the Linux operating system.

    \subsection{Environment Setup: }
    \tabto{20pt}Before we start writing actual code, we need to setup our environment to write, compile and run our code.
    
    For that we need a compiler, we will use nasm (Netwide Assembler).The compiler is the software that will translate the english-like code into machine code stored in an object file (<filename>.o)
    
    Also we will need a linker which assembles all the object files into one single executable file.

    \subsubsection{Linux Environment : }
    \tabto{20pt}On your linux machine you can install nasm through apt (Debian based OS) or pacman (Arch based OS) using the following commands
    \begin{verbatim}
    sudo apt install nasm, binutils
    \end{verbatim}
    for debian based distribution (like Ubuntu, Kali Linux, POP OS\dots)\\
    or instead
    \begin{verbatim}
    sudo pacman -Su nasm, binutils
    sudo pamac install nasm, binutils
    \end{verbatim}
    for Arch based distribution (like Manjaro\dots)
    \subsubsection{Windows Environment : }
    \tabto{20pt}For that we can use Windows subsystem for Linux (WSL 2.0) and a linux terminal.

    - Get the linux terminal from Windows store(windows 10 and better only)
    - Follow the steps specified in \href{https://pureinfotech.com/install-windows-subsystem-linux-2-windows-10/}{this blog}
    \subsection{Your first assembly program: "Hello assmebly"}
    Now that oour environment is completely set up, now it's time to write some assembly code.

    First start by creating a file "let's call it first\_program.asm" (Note that .asm is the extension for assembly program).

    in first\_program.asm write the following code: 

    \begin{verbatim}
        section .data
        message: db "Hello Assmebly", 0xa
        length: equ 15

        section .text
        global _start
        _start:
            MOV EAX, 4 ; output
            MOV EBX, 1 ; standard output
            MOV ECX, message ; string to print
            MOV EDX, length ; number of characters to print
            INT 0x80 ; interrupt to execute the required function

            MOV EAX, 1 ; exit
            MOV EBX, 0 ; exit status
            INT 0x80 
    \end{verbatim}
    Now after writing some code, let's compile and run it.\\
    Head into your terminal, and type the following commands:
    \begin{verbatim}
        nasm -f elf32 first_program.asm -o first_program.o
        ld -m elf_i386 first_program.o -o first_program
        ./first_program
    \end{verbatim}
    And you will in your the terminal
    \begin{verbatim}
        Hello Assembly
    \end{verbatim}
    and the prompt in the next line, meaning that you have successfully written your first working program written in assembly
    \\\textbf{Notes: }\begin{itemize}
        \item Everything in the program will be explained later in the document.
        \item In assembly, there is no escape character (like \textbackslash n) in higher level programming languages, so instead we replace it with its ASCII code $(10)_{10} = (a)_{16}$.
        \item The (0x) prefix determines that the number is in hexadecimal, and (0b) determines that the number is in binary, while a number without a prefix is decimal.
        \item We can replace (0x) prefix by (h) at the end of a hex number, and the (0b) by (b) at the end of a binary number\\
        meaning (0x1cf is identical to 1cfh and 0b11100010111 is identical to 11100010111b).
        \item when using (h) with a hexadecimal number prefix the number with 0 to avoid any errors (ex: ah can be interpreted as the AH register not as 10 in hex but 0ah is the number 10 in hex)
        \item Assembly language is case insensitive, but many developers prefer to write the instructions in uppercase.
        \item In Assembly, we can write one line comments starting with a semicolomn
        \item the first command is the compilation command, the comiler was explained before.
        \item the second command is the linking comand, the linker was previously explained.
        \item the last one is to run the program.
    \end{itemize}
    \tabto{20pt}Now that you've coded and executed your first assembly program, it's time to start exploring assembly instruction set in depth
    \newpage
    \section{Assembly Language in Depth: }
    In this section we will walk through assembly language instructions and how each one of them works.
    \subsection{Memory Sections: }
    The memory organization of today's processors follows an architecture called \textbf{"Von Neueman architecture"}, it states that we use only one memory to store everything, but we split it into sections to better organize it.
    and so that's what we do actually, all the code and data are stored in one RAM but in different locations (specified by the section registers).\\
    \tabto{20pt}And so when we write assembly code we need to specify where this should be stored in the memory, and we can specify two sections only:
    \subsubsection{Data Section: }
    This section contains only the data that the program will use while executing. any variable (initialised or not) and any piece of data will have a reserved chunk of memory there.\\
    \tabto{20pt}When we want to reserve memory for data in memory, we need to specify that we want to store them into data section, give name to the reserved chunk and specify its size.\\
    to specify the chunk of memory we use the following commands:\\
    \begin{verbatim}
        section .data; for initialised variables
        section .bss; for uninitialised variables
    \end{verbatim}
    
    The .data section stores only the data that we already know its values (like "hello assmebly" in our first program).\\
    While .bss section stores only the data that we don't know their values previously.\\

    \subsubsection{How to define data chunks in .data and .bss sections: }
    As mentioned before, we must give the reserved chunk of memory a name, specify the data type (and so the size of the chunk) and the values.\\
    Following is the form for declaring a variable in memory.
    \begin{verbatim}
        <Label>: <type> value
    \end{verbatim}
    \textbf{Label: }specifies the name of the reserved memory chunk.
    \textbf{type: }specifies the type of the variable.\\
    In .data section we must prefix the type with the letter "d" (meaning data).\\
    In .bss section we must prefix the type with the letter "res" (meaning reserve).\\
    And then we give the first character of the type, we can reserve a byte "b", a word (two bytes) "w", a doubleword (four bytes) "d" and a quadword (8 bytes) "q".\\
    After that we can specify the values (if on .data section) seperated by a comma (if we want to declare an "array" of the same type) or the number of chunks to reserve (in .bss section)
    \\\textbf{Example: }
    \begin{verbatim}
    section .data
        data1: dd 5, 3 ,0xc4, 0xff ; an array of 4 double words 
        ;double word is the integer data size in C
        
        data2: db "Assembly is FUN!", 0xa; An array of words 
        
        ;strings are arrays of characters
        ;characters are stored as their ASCII code

        data3: db 256; this is illegal
        ;you get a warning that this exceeds memory size

    section .bss
        var1: resb 5; reserves 5 bytes without initialising them.
        var2: resw 20; reserves 20 words without initialisation.
    \end{verbatim}

    \end{document}